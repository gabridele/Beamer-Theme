\begin{frame}{Discussion}
\setbeamercovered{invisible}
\fontsize{12pt}{16}\selectfont
\only<1>{\begin{itemize}
\item Reignite the hypothesis of a \textbf{coevolution} of tool use and language
\item The \textbf{birth} and \textbf{refinement} of tool use may have offered the \textbf{neural habitat} for the coevolution of both motor and communicative skills.
\end{itemize}}
\pause
\only<2>{\begin{columns}
\hspace{-0.1cm}
\vspace{0.2cm}
    \begin{column}{0.5\textwidth}
        The \textbf{sophistication of tool use} and tool \textbf{making} has put forward the need for cognitive functions to \textbf{efficiently chunk}, temporally \textbf{parse}, and deal with \textbf{hierarchies} of sequences.
    \end{column}
    \begin{column}{0.5\textwidth}
        Tool use and tool making posed \textbf{evolutionary pressure} for communication, allowing better \textbf{social transmission} of knowledge.
    \end{column}
\end{columns}}
\pause
\only<3>{
\begin{itemize}

\item So functions initially serving the motor system would have \textbf{adapted and repurposed} for language and communication.

\item $\rightarrow$ This coevolution scenario has involved a \textbf{broad brain network}: spanning from parietal to frontal regions, including the BG.
\end{itemize}}


\end{frame}