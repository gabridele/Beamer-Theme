\begin{frame}{Discussion}
\setbeamercovered{invisible}
\fontsize{12pt}{16}\selectfont

\only<1>{\begin{itemize}
\item Tool use and syntax rely on brain activity of \textbf{anatomically overlapping} neural networks, particularly in striatal structures (lCau) and the GPi.
\item Working memory processes were not involved.
\end{itemize}}
\pause
\vspace{-0.2cm}
\only<2-3>{{\large\textbf{These findings:\\}}
\begin{itemize}
    \only<2>{\item Support the hypothesis of a \textbf{supramodal syntactic function} serving both domains
    \item \textbf{Consistent} with studies claiming role of the \textbf{dorsal striatum} in \textbf{processing complex hierarchical structures} in both the motor and linguistic domains.
    \item This area is involved in \textbf{syntactic training} and in the implementation of grammatical rules.
    \item It works as a \textbf{parser} of actions to chunk motor sequences}
    \pause
    \only<3>{\item Accurate and efficient tool use requires \textbf{embedding} an \textbf{external object} into the motor sequence and thus relies more on the striatum than on manual actions to parse the motor primitives
    \item \textbf{Parsing} and \textbf{hierarchy} handling also support syntactic comprehension of \textbf{center-embedded object relatives}
    \item These functional similarities are reflected by the neural overlap.}
\end{itemize}
}

\only<4>{\textbf{However,}\\
\begin{itemize}
    \item The complexity of the hierarchies to be handled might not be strictly proportionate [no further elaboration]
    \item \textbf{Overlap} was found in the \textbf{BG} but not in the left IFG
\end{itemize}
}


\end{frame}