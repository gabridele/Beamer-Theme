\begin{frame}{Discussion}
\setbeamercovered{invisible}
\fontsize{12pt}{16}\selectfont
%\vspace{-0.4cm}


\only<1>{
\begin{itemize}
\item Results explain the \textbf{cross-domain learning transfer} from tool use to syntactic skills in language and viceversa.
\item Learning transfer happens only if trained and untrained tasks rely on \textbf{overlapping neural networks} and \textbf{shared cognitive processes}.
\item Transfer effects had been demonstrated so far only in the same domain: perception, motor, or cognitive control.
\end{itemize}}
\pause
\only<2>{
\begin{itemize}
\item Transfer \textbf{holds true} even when different cognitive domains are involved.
\item Transfer might be absent if trained and untrained tasks do not share common neurocognitive resources.
\end{itemize}}
\pause
\only<3>{
\begin{itemize}
\item Training with subject-relative structures did not improve motor performance with the tool.
\item \textbf{Free-hand} training \textbf{failed} to induce benefits to syntax in the comprehension of complex structures.
\end{itemize}}

\pause
\only<4>{
\begin{itemize}
    \item \textbf{Benefits} induced by tool use over language were \textbf{not based} on the mere \textbf{additional sensorimotor complexity} of the action executed with the tool compared with the free hand.
    \item After training with a \textbf{constrained hand}, the participants did not show \textbf{any advantage} in processing complex syntactic structures.
\end{itemize}}
\pause
\only<5>{
\begin{itemize}
    \item The learning \textbf{transfer} between tool use and syntactic processes in language occurs \textbf{bidirectionally}.
    \item Unambiguously indicates that the two abilities rely on a common cognitive component, namely a \textbf{supramodal syntax}.
\end{itemize}}

\end{frame}